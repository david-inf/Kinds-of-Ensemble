% !TeX TS-program = pdflatex
% !TeX encoding = UTF-8
% !TeX spellcheck = en_GB
% !TeX root = exam-main.tex
\documentclass[]{beamer}

\usepackage[T1]{fontenc}
\usepackage{textcomp}
\usepackage[utf8]{inputenc}
\usepackage{babel}

% ------------------ %
%% Presentation settings %%
\usepackage{appendixnumberbeamer}
%\mode<presentation>
\usetheme[progressbar=foot,numbering=counter]{metropolis}
%\useoutertheme[right]{sidebar}
\setbeamercovered{dynamic}
%\setbeamertemplate{blocks}[default]%[shadow=true]
\setbeamertemplate{sections/subsections in toc}[circle]
\setbeamertemplate{items}[circle]
\setbeamertemplate{navigation symbols}{}
\metroset{block=fill}

%\mode<all>
% ------------------ %
%% Floating objects %%
\graphicspath{{../r/plots/},{./Figures/}}
\usepackage{booktabs}
%\usepackage{caption}
%\captionsetup{format=hang,labelsep=colon}%,font={small,rm},labelfont={sf,bf}}
%\captionsetup[table]{skip=-0.9\medskipamount,position=top}
% ------------------ %
%% Mathematical stuffs %%
\usepackage{mathtools}
%\newcommand{\numberset}{\mathbb}
%\newcommand{\R}{\numberset{R}}
\DeclareMathOperator{\sign}{sign}
\DeclareMathOperator*{\argmin}{arg\,min}
%\DeclarePairedDelimiter{\abs}{\lvert}{\rvert}
%\DeclarePairedDelimiter{\norm}{\lVert}{\rVert}
\DeclarePairedDelimiter{\set}{\{}{\}}
%\renewcommand{\epsilon}{\varepsilon}
\renewcommand{\phi}{\varphi}
%\theoremstyle{definition}
%\newtheorem{defs}{Definition}
%\theoremstyle{plain}
%\newtheorem{thm}{Theorem}
% ------------------ %
%% Drawings %%
\usepackage{pgfplots} % + tikz
\pgfplotsset{compat=newest}
\usetikzlibrary{shapes,calc}

\tikzstyle{treenode} = [draw,circle,minimum width=0.6cm,font=\footnotesize,text centered,
inner sep=1.5pt,outer sep=0pt]
\tikzstyle{cloud} = [draw,ellipse,centered,text width=3.5em,text centered,
inner sep=0.5pt,outer sep=0pt,fill=mLightBrown!20,font=\footnotesize]
\tikzstyle{cloud2} = [draw,ellipse,centered,text width=3.2em,text centered,
inner sep=1.5pt,outer sep=0pt,fill=mLightBrown!20,font=\small]
% ------------------ %
%% Code %%
% ------------------ %
%% Numbers %%
\usepackage{siunitx}
\sisetup{%
	output-decimal-marker={.},group-separator={\,},%
	%	round-mode=places,round-precision=6,%
	table-parse-only,table-number-alignment=center%
}
% ------------------------------- %
\usepackage{listings}
\definecolor{Rkwd}{rgb}{0.737,0.353,0.396}
\definecolor{Rparam}{rgb}{0.333,0.667,0.333}
\definecolor{Rnum}{rgb}{0.686,0.059,0.569}
\definecolor{Rstr}{rgb}{0.192,0.494,0.8}
\definecolor{Rcomm}{rgb}{0.678,0.584,0.686}
% R
\lstdefinestyle{Rlang}{language=R,%
	keywordstyle=\color{Rkwd},basicstyle=\small\ttfamily,%
	commentstyle=\color{Rcomm}\ttfamily\em,%
	stringstyle=\color{Rstr}\rmfamily,%
%	numbers=left,numberstyle=\tiny\color{green},stepnumber=1,numbersep=5pt,%
%	showstringspaces=false,breaklines=true,frameround=ftff,%
%	frame=lines,backgroundcolor=\color{lightergray},firstnumber=last,%
	deletekeywords={data,model},%
	morekeywords={TRUE,FALSE,NULL},%
	escapeinside={£!}{!£},%
}
\lstset{style=Rlang}
% ------------------ %
%% Pseudo-code %%
\usepackage[ruled,linesnumbered]{algorithm2e}
\SetNlSty{texttt}{}{}
\SetKw{Or}{or}
\SetKw{And}{and}
\SetKw{Not}{not}
% ------------------ %
%% Printing the presentation %%
%\usepackage{pgfpages}
%\pgfpagesuselayout{4 on 1}[a4paper,border shrink=5mm,landscape]
% ------------------ %
%% Presentation cover %%
%\title{Kinds of Gradient Boosting}
\title{Kinds of Ensembles}
\subtitle{Tested on \dots dataset}
\author{David Nardi}
\date{June 11, 2024}
\institute[UniFi]{MSc in AI, University of Florence}
%\logo{\includegraphics[width=0.2\textwidth]{./Figures/logo}}
%\titlegraphic{\includegraphics{./Figures/logo}}
% ------------------ %
%\usepackage{showframe}
\begin{document}

\pdfbookmark[1]{Title page}{cover}
\maketitle

% ------------------------------- %
% !TeX spellcheck = en_GB

% ------------------------------- %

\begin{frame}{\dots dataset}

\begin{columns}[T]
\begin{column}{0.5\textwidth}

\end{column}
\begin{column}{0.5\textwidth}
Binary classification task (disease -- no disease)

Approaches
\begin{itemize}
	\item 
\end{itemize}
\end{column}
\end{columns}

\end{frame}

% ------------------------------- %
%\appendix
%{\setbeamercolor{palette primary}{fg=black,bg=white}
%\begin{frame}[standout]
%%Thank you for your kindly attention!
%Questions?
%\end{frame}
%}
% ------------------------------- %
% !TeX spellcheck = en_GB

\appendix
\begin{frame}[allowframebreaks]{References}
\begin{thebibliography}{9}
%	\bibitem{goldbach:congettura} Christian Goldbach
%	\newblock Un problema aperto
%	\newblock \emph{Lettera a Leonhard Euler}, 1742
	%% Super learner paper
	\bibitem{sup-learn} E. C. Polley, and M. J. van der Laan
	\newblock Super Learner in Prediction
	\newblock U.C. Berkeley Division of Biostatistics Working Paper Series. Working Paper 266, 2010
\end{thebibliography}
\end{frame}

% !TeX spellcheck = en_GB

\begin{frame}[fragile]{Discrete AdaBoost algorithm}
% stochastic setting
% stochastic gradient boosting framework, when bag.frac < 1
% -> shrinkage added
% -> out-of-bag fraction, on each iteration train a classifier on a different dataset sample, keep the rest as OOB

Discrete AdaBoost with shrinkage and out-of-bag, as an additive model with prediction function $f_m(x)$

{%
\setlength{\interspacetitleruled}{0pt}%
\setlength{\algotitleheightrule}{0pt}%
\begin{algorithm}[H]
\KwIn{$M$, $\set{(x_i,y_i)}_1^N$, $x_i\in\R^p$}
%Initialize observation weights $w_i^{(1)}=1/N$ s.t. $\sum_{i=1}^Nw_i^{(m)}=1$\;
Initialize $f_0(x)=0$\;
\For{$m=1$ \KwTo $M$}{
	Set $w_i^{(m)}=-\frac{\partial L(y,g)}{\partial g}\bigr\rvert_{g=f_m(x)}$ s.t.  $\sum_{i=1}^Nw_i^{(m)}=1$\;
	Fit classifier $G_m(x)$ using $w_i^{(m)}$ with samples from $\pi_m$\;
	Weighted error rate $\text{err}_m=\sum_{i=1}^Nw_i^{(m)}\mathbb{I}(y_i\neq G(x_i))$\;
	Set $\alpha_m=\frac{1}{2}\log\bigl(\frac{1-\text{err}_m}{\text{err}_m}\bigr)$\;
	Update $f_{m}(x)\gets f_{m-1}(x)+\lambda\alpha_mG_m(x)$\;
}
\KwOut{$G(x)=\sign(f_M(x))$}
\end{algorithm}}

\end{frame}

% ------------------------------- %

\begin{frame}{Super Learner algorithm flow diagram}

\begin{figure}
	\includegraphics[width=\textwidth]{./Figures/sup-learn.jpg}
\end{figure}

\end{frame}

\begin{frame}[fragile]%{Super learner algorithm}

{%
\setlength{\interspacetitleruled}{0pt}%
\setlength{\algotitleheightrule}{0pt}%
\begin{algorithm}[H]
\KwIn{$\mathcal{D}=\set{(x_i,y_i)}_1^N$, $\mathcal{L}=\set{\phi_k(X)}_{k=1}^K$}
\ForEach{strong learner in $\mathcal{L}$}{%
	Fit $\phi_k$ on $\mathcal{D}$ $\Rightarrow$ $\hat{\phi}_k(\boldsymbol{X})$
	$\rightarrow$ $\hat{\mathcal{L}}=\set{\hat{\phi}_k}_{k=1}^K$\;
}
%Store the fitted library in $\hat{\mathcal{L}}=\set{\hat{\phi}_k(\boldsymbol{X})}_{k=1}^K$\;
\For{$\nu=1,2,\dots,V$}{%
	\ForEach{strong learner in $\mathcal{L}$}{%
		Fit $\phi_k$ on $T(\nu)$, predict $\hat{\phi}_{k,T(\nu)}(X_i\in V(\nu))$\;
	}
}
Stack output in an $N\times K$ matrix
%$Z=\Bigl\{\hat{\phi}_{k,T(\nu)}\bigl(X_{V(\nu)}\bigr),\,\nu=1,2,\dots,V,\,k=1,2,\dots,K\Bigr\}$\;
$Z=\bigl\{\hat{\phi}_{k,T(\nu)}\bigl(X_{V(\nu)}\bigr)\bigr\}$\;
Propose a family of weighted combinations\vspace{-1em}
\[
m(z\rvert\alpha)=\sum_{k=1}^K\alpha_k\hat{\phi}_{k,T(\nu)}\bigl(\boldsymbol{X}_{V(\nu)}\bigr)
\rightarrow
\hat{\alpha}=\argmin_\alpha\sum_{i=1}^NL(Y_i,m(z_i\rvert\alpha))
\vspace*{-1em}\]
of size $N$ s.t. $\alpha_k\geq0$, $\sum_k\alpha_k=1$ and minimizes $\sum_k\alpha_k\hat{\phi}_k$\;
Combine $\hat{\alpha}$ with the library $\hat{\mathcal{L}}$ $\rightarrow$ 
$\hat{\phi}_{\text{SL}}(\boldsymbol{X})=\sum_{k=1}^K\hat{\alpha}_k\hat{\phi}_k(\boldsymbol{X})$\;
\KwOut{$\hat{\phi}_{\text{SL}}$}
\end{algorithm}}

\end{frame}


\end{document}
